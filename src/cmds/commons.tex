\usepackage{calculator}

%%%%%%%%%%%%%%%%%%%%%%%%%%%%%%%%%%%%%%%%%%%%%
% REFERENCES
%%%%%%%%%%%%%%%%%%%%%%%%%%%%%%%%%%%%%%%%%%%%%

\makeatletter

% Add a reference to a section
% param1 the label of the section
\NewDocumentCommand{\lstref}{m}{%
    Listing \ref{#1}%
}

% Add a reference to a section
% param1 the label of the section
\NewDocumentCommand{\listingref}{m}{%
    Listing \ref{#1}%
}


% Add a reference to a subsubsection
% param1 the label of the subsubsection
\NewDocumentCommand{\subsubsectionref}{m}{%
    Sub Sub Section \ref{#1}%
}

% Add a reference to a section
% param1 the label of the section
\NewDocumentCommand{\sectionref}{m}{%
    Section \ref{#1}%
}

% Add a reference to a equation
% param1 the label of the equation
\NewDocumentCommand{\equationref}{m}{%
    Equation \eqref{#1}%
}

% Add a reference to a algorithm line
% param1 the label of the algorithm line
\NewDocumentCommand{\alglineref}{m}{%
    Line \ref{#1}%
}

% Add a reference to a algorithm line
% param1 the label of the algorithm where piece starts
% param1 the label of the algorithm where priece ends
\NewDocumentCommand{\alglinesref}{m m}{%
    Lines \ref{#1}--\ref{#2}%
}

% Add a reference to an algorithm
% param1 the label of the algorithm
\NewDocumentCommand{\algref}{m}{%
    Algorithm \ref{#1}%
}

% Add a reference to a definition
% param1 the label of the definition
\NewDocumentCommand{\defref}{m}{%
    Definition \ref{#1}%
}

% Add a reference to a definition
% param1 the label of the first definition
% param2 the label of the second definition
\NewDocumentCommand{\defsref}{m m}{%
    Definitions \ref{#1}--\ref{#2}%
}

% Add a reference to a figure
% param1 the label of the figure
\NewDocumentCommand{\figref}{m}{%
    Figure \ref{#1}%
}

% Add a reference to a range of figure
% param1 the label of the first figure in the range (inclusive)
% param2 the label of the last figure in the range (inclusive)
\NewDocumentCommand{\figsref}{m m}{%
    Figures \ref{#1}--\ref{#2}%
}


\NewDocumentCommand{\r@figrefs}{m}{%
    \@ifnextchar\bgroup{, \ref{#1}\r@figrefs}{ and \ref{#1}}%
}
% Add a reference to a figure
% param1 the label of the figure
\NewDocumentCommand{\figrefs}{m}{%
    Figures \ref{#1}\r@figrefs%
}

% Add a reference to a table
% param1 the label of the table
\NewDocumentCommand{\tblref}{m}{%
    Table \ref{#1}%
}

% Add a reference to a theorem
% param1 the label of the table
\NewDocumentCommand{\thmref}{m}{%
    Theorem \ref{#1}%
}

% Add a reference to a lemma
% param1 the label of the lemma
\NewDocumentCommand{\lemmaref}{m}{%
    Lemma \ref{#1}%
}

% Add a reference to an example
% param1 the label of the example
\NewDocumentCommand{\exampleref}{m}{%
    Example \ref{#1}%
}

%%%%%%%%%%%%%%%%%%%%%%%%%%%%%%%%%%%%%%%%%%%%%
% EXAMPLE
%%%%%%%%%%%%%%%%%%%%%%%%%%%%%%%%%%%%%%%%%%%%%

\theoremstyle{definition}
\newtheorem{example}{Example}[section]

%%%%%%%%%%%%%%%%%%%%%%%%%%%%%%%%%%%%%%%%%%%%%
% USEFUL COMMANDS
%%%%%%%%%%%%%%%%%%%%%%%%%%%%%%%%%%%%%%%%%%%%%

%DOES NOT WORK
\NewDocumentEnvironment{verticalAlign}{+b}{%
    \topskip0pt%
    \vspace*{\fill}%
    {#1}%
    \vspace*{\fill}%
}{%
}

% Indent a body of text
% param1 dimension representing the space you want to indent
% param2 body actual content to put in the indented wall of text
\NewDocumentEnvironment{indentText}{O{3cm} +b}{%
    \begin{minipage}{\dimexpr\textwidth-#1}%
        #2%
        \xdef\tpd{\the\prevdepth}%
    \end{minipage}%
}{%
}

% Put a definition block in the text
%
% param1 the definition name. If not specified we won't have a definition
% param2 the label we want this definition to have. Input of \label. If not specified we won't put the \label command
% param3 the content of the definition
% \NewDocumentEnvironment{definition}{o o +b}{%
%     %BEGIN
%     \IfNoValueTF{#1}{%
%         \begin{theorem}%
%     }{%
%         \begin{theorem}[#1]%
%     }%
%     \IfNoValueF{#2}{%
%         \label{#2}%
%     }%
%     #2%
%     \end{theorem}%
% }{%
%     %END
% }
%     \IfNoValueTF{#1}{%
%         \begin{theorem}
%     }{%
%     }%
% }

\NewDocumentCommand{\setFontSize}{m o m}{%
    \IfNoValueF{#2}{%
        \fontsize{#1}{#2}\selectfont#3%
    }{%
        % see https://texblog.org/2012/08/29/changing-the-font-size-in-latex/
        \MULTIPLY{#1}{1.2}{\setFont@baseline}%
        \fontsize{#1}{\setFont@baseline}\selectfont#3%
    }%
}

% Adds a todo embedded in the text (colored blue)
% param1: an optional star: if a star is present, we will put the todo as a footnote
% param2: the text to put
\NewDocumentCommand{\todo}{s m}{%
    \IfBooleanTF{#1}{%
        \footnote{\color{blue} #2}%
    }{%
        {\color{blue} #2}%
    }%
}

%Adds a todo as a footnoe
\NewDocumentCommand{\code}{m}{%
    \texttt{#1}%
}

% draw a square.
% param1: fill color red!50
% param2: border color (default to black)
\NewDocumentCommand{\drawFilledSquare}{m O{black}}{%
    \begin{tikzpicture}%
        \node [rectangle,draw={#2},fill={#1}] (m) at (0,0) {};%
    \end{tikzpicture}%
}

% print a computer science ordered pair
% param1 first element of the pair
% param2 second element of the pair
\NewDocumentCommand{\paircs}{m m}{%
    \wrapMath{\langle {#1}, {#2} \rangle }%
}

\NewDocumentEnvironment{coloredBlock}{m O{blue} O{white}}{%
\setbeamercolor{block title}{bg=#2, fg=#3}
    \begin{block}{#1}%
}{%
    \end{block}%
}

\NewDocumentCommand{\stacksymbols}{m m}{%
    \wrapMath{\stackrel{\mathclap{#1}}{#2}}%
}

\NewDocumentCommand{\bigO}{m}{%
    \wrapMath{\mathcal{O}(#1)}%
}

\NewDocumentCommand{\nil}{}{%
    \texttt{NIL}%
}

\NewDocumentCommand{\doublePlus}{}{%
    \ifmmode{+\!\!+}\else{$+\!\!+$}\fi%
}

\NewDocumentCommand{\isInMath}{m m}{%
    \ifmmode{#1}\else{#2}\fi%
}

\NewDocumentCommand{\wrapMath}{m}{%
    \ifmmode{#1}\else{$#1$}\fi%
}

%apply double quotes on the parameter
% param1 the text to wrap quote
\NewDocumentCommand{\dquote}{m}{%
    ``{#1}''%
}

\NewDocumentCommand{\squote}{m}{%
    \isInMath%
        {\mbox{`}{#1}\mbox{'}}%
        {`{#1}'}%
}

%%%%%%%%%%%%%%%%%%%%%%%%%%%%%%%%%%%%%%%%%%%%%
% SYMBOLS
%%%%%%%%%%%%%%%%%%%%%%%%%%%%%%%%%%%%%%%%%%%%%

% draw a "v" representing a checkbox which has been checked
\NewDocumentCommand{\checked}{}{%
\tikz\fill[scale=0.4](0,.35) -- (.25,0) -- (1,.7) -- (.25,.15) -- cycle;%
}

% draw a "x" representing a checkbox which has been checked
\NewDocumentCommand{\unchecked}{}{%
\tikz\fill[scale=0.4]%
    (-0.35,+0.35) -- (+0.00,+0.07) --%
    (+0.40,+0.40) -- (+0.07,+0.00) --%
    (+0.35,-0.35) -- (+0.00,-0.07) --%
    (-0.40,-0.40) -- (-0.07,+0.00) --%
    cycle;%
}

%%%%%%%%%%%%%%%%%%%%%%%%%%%%%%%%%%%%%%%%%%%%%
% ACRONYM
%%%%%%%%%%%%%%%%%%%%%%%%%%%%%%%%%%%%%%%%%%%%%

\NewDocumentCommand{\eg}{}{%
    e.g.,%
}

\NewDocumentCommand{\ie}{}{%
    i.e.,%
}

\NewDocumentCommand{\st}{}{%
    s.t.%
}

\NewDocumentCommand{\wrt}{}{%
    w.r.t.%
}
\RenewDocumentCommand{\iff}{}{%
    \textit{iff}%
}

% see https://tex.stackexchange.com/a/369691/145331
\let\@oldcite\cite
\renewcommand*\cite[1]{~\@oldcite{#1}}


\makeatother
